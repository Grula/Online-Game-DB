\documentclass{article}
\usepackage[utf8]{inputenc}
\usepackage[T1]{fontenc}

\begin{document}

\section {Opis}
TODO: Promeniti tekst
Za korisnika(player) se cuvaju informacije o username-u,email-u,password-u i nickname-u. Jedan korisnika(player) moze da ima 0 ili vise prijatelja(friends)- drugih korisnika.
Za korisnika se cuva login\_history, moze da postoji 0 ili vise logova u zavisnosti koliko se player logovao u sistem, login\_history ne postoji ako ne postoji korisnik.
Korisnik moze da napravi vise instanci neke igre, a i ne mora ni jednu.

Level oznacava o kom se levelu radi, i sadrzi samo jedan atribut id. Player sa levelom pravi instancu jedne igre, level moze da ucestvuje u vise instanci a ne mora ni u jednoj.
Level ucestvuje u vise mapa, ali mora bar u jednoj. Za svaki level postoje oponent-i koje on sadrzi, moze da ima vise oponenta i mora da ima bar jednog.

Za svakog oponent-a se cuva id i njegovo ime. Svaki oponent koristi jedan ili vise skill-ova. On moze da ucestvuje u vise levela ali i ne mroa ni u jednom.

Character sadrzi id i svoje ime. Character koristi neki skill, moze da koristi vise ali mora da ima bar jedan. Character moze da bude u vise instanci igre ali i ne mora ni u jednoj. Game instanca moze da ima vise charactera ali mora da ima bar jednog.

Mapa cuva svoj id, na kom se levelu nalazi, i kojoj game\_instaci, kad je napravaljena i koliko poena je korisnik osvojio na njoj. Mapa je sasavljenja od vise objekata ali barem jednog.

Objekat sadrzi svoj id, ime, i putanju do slike koju koristi. Objekat moze da ucestvuje u mapi, moze da ucestvuje vise puta ali ni ne mora.

\section {Uslovi}

\begin{itemize}
\item Nezavisni entiteti
	\begin{enumerate}
	\item Player - Korisnik koji ima napravljen nalog u nasoj igri.
	\item Object - Objekat koji se koristi pri pravljenju mape.
	\item Character - Korisnig bira neke od postojecih likova i koristi ih u igri.
	\item Skill - Ponudjene vestine koje korisnik moze da dodeli nekim od likova koje je izabrao.
	\item Oponenet - NPC koji se pojavljuju na mapi.
	\item Level - Sadrzi samo id i govori o kojem je nivou rec.
	\end{enumerate}

\item Agregirani entiteti
	\begin{enumerate}
	\item Game\_instance - Opisuje na kom levelu, na kojoj mapi i kog lika korisnik koristi.
	\item Character\_used - Opisuje koji se od likova koristi u instanci igre.
	\item Character\_has\_skill - Opisuje koji koje vestine nas lik koristi.
	\item Oponent\_has\_skill - Opisuje koje vestine imaju protivnici.
	\item Level\_has\_oponent - Opisuje koje protivnike ima trenutni nivo.
	\item Map\_has\_object - Opisuje koje vestine imaju protivnici.

	\end{enumerate}

\item Slab entitet
	\begin{enumerate}
	\item login\_history - Cuva ifnormacije kada se korisnik logovao u igru.
	\end{enumerate}

\item Rekurzivni odnos
	\begin{enumerate}
	\item Player\_friends - Cuva informaciju koji player je prijatelj sa kojim player-om.
	\end{enumerate}

\item Trigeri
	\begin{enumerate}
	\item triger 1 \newline
		Lorem ipsum dolor sit amet, consectetur adipiscing elit. Aenean vestibulum leo vel facilisis lacinia. Cras gravida nisl est. \newline
	
	\item triger 1 \newline
		Lorem ipsum dolor sit amet, consectetur adipiscing elit. Aenean vestibulum leo vel facilisis lacinia. 

	\end{enumerate}

\end{itemize}


\end{document}
