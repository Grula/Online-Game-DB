\documentclass{article}
\usepackage[utf8]{inputenc}
\usepackage[T1]{fontenc}

\begin{document}
\section {Uslovi}

\begin{itemize}
\item Nezavisni entiteti
	\begin{enumerate}
	\item Player - Korisnik koji ima napravljen nalog u nasoj igri.
	\item Object - Objekat koji se koristi pri pravljenju mape.
	\item Character - Korisnig bira neke od postojecih likova i koristi ih u igri.
	\item Skill - Ponudjene vestine koje korisnik moze da dodeli nekim od likova koje je izabrao.
	\item Oponenet - NPC koji se pojavljuju na mapi.
	\item Level - Sadrzi samo id i govori o kojem je nivou rec.
	\end{enumerate}

\item Agregirani entiteti
	\begin{enumerate}
	\item Character\_used - Opisuje koji se od likova koristi u instanci igre.
	\item Character\_has\_skill - Opisuje koji koje vestine nas lik koristi.
	\item Oponent\_has\_skill - Opisuje koje vestine imaju protivnici.
	\item Level\_has\_oponent - Opisuje koje protivnike ima trenutni nivo.
	\item Map\_has\_object - Opisuje koje vestine imaju protivnici.

	\end{enumerate}

\item Slab entitet
	\begin{enumerate}
	\item login\_history - Cuva ifnormacije kada se korisnik logovao u igru.
	\end{enumerate}

\item Rekurzivni odnos
	\begin{enumerate}
	\item Player\_friends - Cuva informaciju koji player je prijatelj sa kojim player-om.
	\end{enumerate}

\item Trigeri
	\begin{enumerate}
	\item ...
	\end{enumerate}

\end{itemize}


\end{document}
